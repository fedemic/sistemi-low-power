\documentclass[10pt,  english, makeidx, a4paper, titlepage, oneside]{book}

% Packages
\usepackage{graphicx} % Usato per \includegraphics 
\usepackage{titlesec} % Usato per modificare/rimuovere le scritte Capitolo 1,2 ecc
%\usepackage[sfdefault]{Roboto}  
%\usepackage[T1]{fontenc}

% Impostazioni pagina
\textwidth 15.5cm % larghezza del testo nella pagina
\textheight 23cm % altezza del testo nella pagina
\topmargin -2cm % spazio in più o in meno dall'alto
\oddsidemargin 0cm % si può spostare il margine dove inizia il testo nella pagina 
\titleformat{\chapter}[display]
{\normalfont\bfseries}{}{0pt}{\Huge} % Rimuove le scritte Capitolo 1,2 ecc




% inizio del documento
\begin{document}

% pagina iniziale
\begin{titlepage}

	\centerline{\includegraphics[width=3cm]{./img/general/polito.png}}
	\vspace{0.3cm}
	\centerline{\Large{Politecnico di Torino}}
	\vspace{0.3cm}
	\centerline{\Large{III Facoltà di Ingegneria}}
	\vspace{3cm}
	\centerline{\Huge\textbf{Sistemi elettronici a basso consumo}}
	\vspace{1cm}
	\centerline{\LARGE\textbf{Relazioni di laboratorio}}
	\vspace{3cm}
	\centerline{\LARGE{Laurea Magistrale in Ingegneria Elettronica}}
	\vspace{0.3cm}
	\centerline{\LARGE{Orientamento: Sistemi Elettronici}}
	\vspace{3cm}
	\centerline{\Large{Gruppo n. 9}}
	\vspace{2cm}
	\centerline{\Large{Autori:}}
	\vspace{0.3cm}
	\centerline{\Large{Favero Simone, Micelli Federico, Spanna Francesca}}
	
\end{titlepage}

\tableofcontents % imposta l'indice in automatico
\pagebreak % per passare direttamente alla prossima pagina

\chapter{Laboratorio 1}


\section{Calcolo di probabilità e attività: porte logiche elementari}

Il primo esercizio consiste nel valutare le probabilità e attività di 
quattro porte logiche elementari: NOT, AND, OR e XOR.
\\
Mentre la probabilità di uscita del gate è definita dalla funzione logica
stessa, la switching activity è valutata allo stesso modo per tutti i casi, 
mediante la seguente formula:
\\\\
	\centerline{$A = 2 \cdot P1 \cdot (1-P0)$}
\\\\
Di seguito è riportata l'analisi delle porte logiche richieste, considerando 
ingressi equiprobabili e scorrelati.

\begin{itemize}
	\item \textbf{NOT} \\
	$P(Y=1) = 1 - P(A=1) = 0.5$ \\
	$A(Y) = 0.5$
	\item \textbf{AND} \\
	$P(Y=1) = P(A=1) \cdot P(B=1) = 0.25$ \\
	$A(Y) = 0.375$
	\item \textbf{OR} \\
	$P(Y=1) = 1-((1-P(A=1)) \cdot(1-P(B=1))) = 0.75$ \\
	$A(Y) = 0.375$
	\item \textbf{XOR} \\
	$P(Y=1) = P(A=1) \cdot(1-P(B=1)) + P(B=1) \cdot(1-P(A=1)) = 0.5$ \\
	$A(Y) = 0.5$ 
\end{itemize} 
Simulando il test bench fornito tramite ModelSim è possibile ottenere un file 
riportante il numero di commutazioni di ogni segnale del circuito durante il 
tempo di simulazione.
\\
Il testbench fornito sfrutta un generatore di numeri casuali per generare gli 
ingressi delle porte, rendendo questi ultimi equiprobabili e statisticamente 
indipendenti.
\\
In particolare, è possibile ricavare la switching activity delle uscite dividendo 
il numero di commutazioni per il numero di cicli di clock simulati.
\\
Sono riportati i seguenti valori:
\\
\begin{center}
	\begin{tabular}{|c|c|c|c|c|} %numero di colonne
 	\hline % tira una riga
 	Tc(CK) & Tc(INV) & Tc(AND) & Tc(OR) & Tc(XOR) \\ 
 	\hline
 	20 & 1 & 0 & 4 & 4 \\ 
 	\hline
 	200 & 43 & 40 & 42 & 44 \\ 
 	\hline
 	2000 & 533 & 418 & 352 & 470 \\
 	\hline
 	20000 & 4916 & 3606 & 3784 & 4876 \\
 	\hline
 	200000 & 49967 & 37834 & 37541 & 49939 \\
 	\hline
	\end{tabular}
\end{center}
E' possibile stimare la switching activty dividendo il numero 
di commutazioni di un nodo per il numero di colpi di clock della
relativa simulazione.
\\
Dal momento che il parametro Tc si riferisce al numero totale di 
commutazioni, il numero di cicli di clock è ottenuto dividendo per due
tale parametro.
\\
I risultati dei calcoli sono riportati nella seguente tabella. 
\\
\begin{center}
	\begin{tabular}{|c|c|c|c|c|}
	\hline
	Tc(CK) & Tc(INV) & Tc(AND) & Tc(OR) & Tc(XOR) \\ 
	\hline
	20 & 0.1 & 0 & 0.4 & 0.4 \\
	\hline
	200 & 0.43 & 0.40 & 0.42 & 0.44 \\
	\hline
	2000 & 0.533 & 0.418 & 0.352 & 0.470 \\
	\hline
	20000 & 0.4916 & 0.3606 & 0.3784 & 0.4876 \\
	\hline
	200000 & 0.4997 & 0.3738 & 0.3754 & 0.4939 \\
	\hline
	\end{tabular}
\end{center}
\vspace{0.3cm}
Per garantire una migliore visualizzazione dei dati ottenuti 
al variare del tempo di simulazione, sono stati realizzati 
i seguenti grafici.
\\\\\\
\centerline{\includegraphics[width=15cm]{./img/Lab_1/Es_1/Grafici_Es_1.png}}
\\\\\\
Si osserva che all'aumentare del tempo di simulazione la stima 
dell'attività risulta a man mano più accurata. In particolare, 
nel caso analizzato, si osserva che per un numero di cicli di clock 
superiore a 10000, i dati sono confrontabili con quelli teorici.
\pagebreak

\section{Calcolo di probabilità e attività: half adder e full adder}

Dalle tavole di verità di Half Adder e Full Adder si ottengono le seguenti funzioni:
\\
\begin{itemize}
	\item \textbf{Half adder} \\
	S = A XOR B \\
	Cout = A AND B
	\item \textbf{Full adder} \\
	S = A XOR B XOR Cin \\
	Cout = A AND B AND Cin \\
\end{itemize}
Partendo dalle funzioni delle uscite è stato possibile ricavare le probabilità
associate alle uscite e le relative attività.
\\
\begin{itemize}
	\item \textbf{Half adder}\\\\
	$P(S=1) = P(A=1) \cdot ((1-P(B=1)) + P(B=1) \cdot (1-P(A=1))$ \\
	$P(Cout=1) = P(A=1) \cdot P(B=1)$ \\
	$A(S) = 2 \cdot P(S=1) \cdot (1-P(S=1))$ \\
	$A(Cout) = 2 \cdot P(Cout=1) \cdot (1-P(Cout=1))$ \\
	\item \textbf{Full adder} \\\\
	$P(S=1) = P(A=1) \cdot(1-P(B=1)) \cdot (1-P(Cin=1)) + \\ 
	          P(B=1) \cdot(1-P(A=1)) \cdot (1-P(Cin=1)) + \\
	          P(Cin=1) \cdot (1-P(A=1)) \cdot(1-P(B=1)) + \\
	          P(A=1) \cdot P(B=1) \cdot P(Cin=1)$	          
	$P(Cout=1) =$
	$A(S) =$
	$A(COut) =$
\end{itemize} 
\pagebreak




\end{document}
