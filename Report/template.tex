\documentclass[12pt,  english, makeidx, a4paper, titlepage, oneside]{article}
\usepackage{babel}
\usepackage{fancyhdr}
\usepackage{makeidx}
\usepackage{titlesec}
\usepackage{listings} 
\newenvironment{listato}{\footnotesize}
                        {\normalsize }
\textwidth 16cm
\textheight 23cm
\topmargin -1.5cm
\oddsidemargin -0cm
\linespread{1.2}

\pagestyle{fancy}
\lhead{}
\chead{}
\lfoot{}
\cfoot{}
\rfoot{}
\rhead{\thepage}

\usepackage{graphicx}
\usepackage{amsmath}
\usepackage{amsfonts}
\usepackage{amsthm}
\usepackage{amssymb}

%%%%modified for listing code
\usepackage{color}
\definecolor{codegreen}{rgb}{0,0.6,0}
\definecolor{codegray}{rgb}{0.5,0.5,0.5}
\definecolor{codepurple}{rgb}{0.58,0,0.82}
\definecolor{backcolour}{rgb}{0.95,0.95,0.92}

\usepackage{hyperref}
\usepackage{booktabs} 
\usepackage{colortbl} 
\usepackage{xcolor} 
\usepackage{xfrac}



\usepackage{float} %%to fix images
\usepackage{graphicx}
\usepackage{subfigure}
\usepackage{subfloat}
\usepackage{caption}
%SI reference
\usepackage{siunitx}

%images on tabular in correct way
\usepackage{array}
\newcolumntype{P}[1]{>{\centering\arraybackslash}p{#1}}
\usepackage{mwe}
\usepackage{graphbox}

%to draw electrical circuits
\usepackage{tikz}
\usepackage[americanresistors,americaninductors]{circuitikz}
%%%%%%%%%%%%%%%%%%%%%%%%%%%%%%%%%

\titleformat{\chapter}[display]
{\normalfont\Large\filcenter\sffamily}
{\titlerule[0.5pt]%
\vspace{1pt}
\titlerule
\vspace{1pc}
\LARGE\MakeUppercase{\chaptertitlename} \thechapter
}
{1pc}
{\titlerule
\vspace{1pc}
\Huge}

\makeindex

\begin{document}
\begin{titlepage}
\centerline{\Large\textbf{Signal processing} }
\vspace{0.3cm}
\centerline{\Large\textbf{and} }
\vspace{0.3cm}
\centerline{\Large\textbf{Wireless trasmission lab} }
\bigskip
\bigskip
\bigskip
\centerline{
\includegraphics[width=4cm]{./image/polito.png}}  
\vspace{1cm}
\centerline{\LARGE Politecnico di Torino}
%\centerline{\Large III Facolt\`a di Ingegneria}
\vspace{2cm}
\centerline{\Huge\sf Wireless relation }
\bigskip
\bigskip

\vspace{2cm}
\centerline{\Large Colonna Antonio 277965}
\vspace{0.5cm}
\centerline{\Large  Giuffrida Luigi 278039 }
\vspace{0.5cm}
\centerline{\Large Spanna Francesca 278040}
\vspace{0.5cm}
\centerline{\Large Simonetti Marco 277994}
\vspace{0.5cm}
\centerline{\Large GROUP-10}
\vspace{3.25cm}
\centerline{\Large ACADEMIC YEAR: 2019-2020}
\end{titlepage}

\tableofcontents

\pagebreak

\section{Introduction}
Wireless transmission is experimental course based on the experience improved in laboratory.
In the first part theoretical topics were addressed with regard to:
\begin{itemize}
\item Overview on the main components of the physical layer of broadband fixed and mobile radio-communication standards.
\item Structure and main functions of conventional and software defined digital radio transceivers.
\item Software defined radio (SDR) platforms.
\end{itemize}

Instead, the second and most important part, is based on
\begin{itemize}
\item Implementation on SDR platform (Ettus B210 and National Instruments USRP-2901) of some common transceiver blocks suitable for most wireless communication standards (WiFi, LTE, DVB)
\item Test and measurements of the performance of some implemented receiver algorithms
\end{itemize}

The main purpose in laboratory is to implement on suitable hardware/software platforms some of the blocks of the studied wireless communication standards.
To do this, the software "Visual Studio" based on the C++ language has been used and the library, developed by the Polytechnic of Turin, called "TOPCOM++" has been used.

\section{First laboratory}
Running a simulation whit TOPCOM++ implies the usage of different classes and fundamental blocks that implements the behaviour of a real communication system; the easiest one is composed by a transmitter, a channel and a receiver, in particular the channel, to make the simulation more realistic adds a white gaussian noise to the signal the is running through 


\section{Second laboratory}
\subsection{Goals}
Purpose of this Lab is to analyze the system by producing plots with different parameters.
The parameters th


\begin{figure}[H]
    \centering
    \includegraphics[width=1.0\columnwidth]{image/without_filter.PNG}
    \caption*{System without TX and RX filters}
    \label{fig:without_filter}
\end{figure}

\end{document}
